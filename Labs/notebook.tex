
% Default to the notebook output style

    


% Inherit from the specified cell style.




    
\documentclass[11pt]{article}

    
    
    \usepackage[T1]{fontenc}
    % Nicer default font (+ math font) than Computer Modern for most use cases
    \usepackage{mathpazo}

    % Basic figure setup, for now with no caption control since it's done
    % automatically by Pandoc (which extracts ![](path) syntax from Markdown).
    \usepackage{graphicx}
    % We will generate all images so they have a width \maxwidth. This means
    % that they will get their normal width if they fit onto the page, but
    % are scaled down if they would overflow the margins.
    \makeatletter
    \def\maxwidth{\ifdim\Gin@nat@width>\linewidth\linewidth
    \else\Gin@nat@width\fi}
    \makeatother
    \let\Oldincludegraphics\includegraphics
    % Set max figure width to be 80% of text width, for now hardcoded.
    \renewcommand{\includegraphics}[1]{\Oldincludegraphics[width=.8\maxwidth]{#1}}
    % Ensure that by default, figures have no caption (until we provide a
    % proper Figure object with a Caption API and a way to capture that
    % in the conversion process - todo).
    \usepackage{caption}
    \DeclareCaptionLabelFormat{nolabel}{}
    \captionsetup{labelformat=nolabel}

    \usepackage{adjustbox} % Used to constrain images to a maximum size 
    \usepackage{xcolor} % Allow colors to be defined
    \usepackage{enumerate} % Needed for markdown enumerations to work
    \usepackage{geometry} % Used to adjust the document margins
    \usepackage{amsmath} % Equations
    \usepackage{amssymb} % Equations
    \usepackage{textcomp} % defines textquotesingle
    % Hack from http://tex.stackexchange.com/a/47451/13684:
    \AtBeginDocument{%
        \def\PYZsq{\textquotesingle}% Upright quotes in Pygmentized code
    }
    \usepackage{upquote} % Upright quotes for verbatim code
    \usepackage{eurosym} % defines \euro
    \usepackage[mathletters]{ucs} % Extended unicode (utf-8) support
    \usepackage[utf8x]{inputenc} % Allow utf-8 characters in the tex document
    \usepackage{fancyvrb} % verbatim replacement that allows latex
    \usepackage{grffile} % extends the file name processing of package graphics 
                         % to support a larger range 
    % The hyperref package gives us a pdf with properly built
    % internal navigation ('pdf bookmarks' for the table of contents,
    % internal cross-reference links, web links for URLs, etc.)
    \usepackage{hyperref}
    \usepackage{longtable} % longtable support required by pandoc >1.10
    \usepackage{booktabs}  % table support for pandoc > 1.12.2
    \usepackage[inline]{enumitem} % IRkernel/repr support (it uses the enumerate* environment)
    \usepackage[normalem]{ulem} % ulem is needed to support strikethroughs (\sout)
                                % normalem makes italics be italics, not underlines
    

    
    
    % Colors for the hyperref package
    \definecolor{urlcolor}{rgb}{0,.145,.698}
    \definecolor{linkcolor}{rgb}{.71,0.21,0.01}
    \definecolor{citecolor}{rgb}{.12,.54,.11}

    % ANSI colors
    \definecolor{ansi-black}{HTML}{3E424D}
    \definecolor{ansi-black-intense}{HTML}{282C36}
    \definecolor{ansi-red}{HTML}{E75C58}
    \definecolor{ansi-red-intense}{HTML}{B22B31}
    \definecolor{ansi-green}{HTML}{00A250}
    \definecolor{ansi-green-intense}{HTML}{007427}
    \definecolor{ansi-yellow}{HTML}{DDB62B}
    \definecolor{ansi-yellow-intense}{HTML}{B27D12}
    \definecolor{ansi-blue}{HTML}{208FFB}
    \definecolor{ansi-blue-intense}{HTML}{0065CA}
    \definecolor{ansi-magenta}{HTML}{D160C4}
    \definecolor{ansi-magenta-intense}{HTML}{A03196}
    \definecolor{ansi-cyan}{HTML}{60C6C8}
    \definecolor{ansi-cyan-intense}{HTML}{258F8F}
    \definecolor{ansi-white}{HTML}{C5C1B4}
    \definecolor{ansi-white-intense}{HTML}{A1A6B2}

    % commands and environments needed by pandoc snippets
    % extracted from the output of `pandoc -s`
    \providecommand{\tightlist}{%
      \setlength{\itemsep}{0pt}\setlength{\parskip}{0pt}}
    \DefineVerbatimEnvironment{Highlighting}{Verbatim}{commandchars=\\\{\}}
    % Add ',fontsize=\small' for more characters per line
    \newenvironment{Shaded}{}{}
    \newcommand{\KeywordTok}[1]{\textcolor[rgb]{0.00,0.44,0.13}{\textbf{{#1}}}}
    \newcommand{\DataTypeTok}[1]{\textcolor[rgb]{0.56,0.13,0.00}{{#1}}}
    \newcommand{\DecValTok}[1]{\textcolor[rgb]{0.25,0.63,0.44}{{#1}}}
    \newcommand{\BaseNTok}[1]{\textcolor[rgb]{0.25,0.63,0.44}{{#1}}}
    \newcommand{\FloatTok}[1]{\textcolor[rgb]{0.25,0.63,0.44}{{#1}}}
    \newcommand{\CharTok}[1]{\textcolor[rgb]{0.25,0.44,0.63}{{#1}}}
    \newcommand{\StringTok}[1]{\textcolor[rgb]{0.25,0.44,0.63}{{#1}}}
    \newcommand{\CommentTok}[1]{\textcolor[rgb]{0.38,0.63,0.69}{\textit{{#1}}}}
    \newcommand{\OtherTok}[1]{\textcolor[rgb]{0.00,0.44,0.13}{{#1}}}
    \newcommand{\AlertTok}[1]{\textcolor[rgb]{1.00,0.00,0.00}{\textbf{{#1}}}}
    \newcommand{\FunctionTok}[1]{\textcolor[rgb]{0.02,0.16,0.49}{{#1}}}
    \newcommand{\RegionMarkerTok}[1]{{#1}}
    \newcommand{\ErrorTok}[1]{\textcolor[rgb]{1.00,0.00,0.00}{\textbf{{#1}}}}
    \newcommand{\NormalTok}[1]{{#1}}
    
    % Additional commands for more recent versions of Pandoc
    \newcommand{\ConstantTok}[1]{\textcolor[rgb]{0.53,0.00,0.00}{{#1}}}
    \newcommand{\SpecialCharTok}[1]{\textcolor[rgb]{0.25,0.44,0.63}{{#1}}}
    \newcommand{\VerbatimStringTok}[1]{\textcolor[rgb]{0.25,0.44,0.63}{{#1}}}
    \newcommand{\SpecialStringTok}[1]{\textcolor[rgb]{0.73,0.40,0.53}{{#1}}}
    \newcommand{\ImportTok}[1]{{#1}}
    \newcommand{\DocumentationTok}[1]{\textcolor[rgb]{0.73,0.13,0.13}{\textit{{#1}}}}
    \newcommand{\AnnotationTok}[1]{\textcolor[rgb]{0.38,0.63,0.69}{\textbf{\textit{{#1}}}}}
    \newcommand{\CommentVarTok}[1]{\textcolor[rgb]{0.38,0.63,0.69}{\textbf{\textit{{#1}}}}}
    \newcommand{\VariableTok}[1]{\textcolor[rgb]{0.10,0.09,0.49}{{#1}}}
    \newcommand{\ControlFlowTok}[1]{\textcolor[rgb]{0.00,0.44,0.13}{\textbf{{#1}}}}
    \newcommand{\OperatorTok}[1]{\textcolor[rgb]{0.40,0.40,0.40}{{#1}}}
    \newcommand{\BuiltInTok}[1]{{#1}}
    \newcommand{\ExtensionTok}[1]{{#1}}
    \newcommand{\PreprocessorTok}[1]{\textcolor[rgb]{0.74,0.48,0.00}{{#1}}}
    \newcommand{\AttributeTok}[1]{\textcolor[rgb]{0.49,0.56,0.16}{{#1}}}
    \newcommand{\InformationTok}[1]{\textcolor[rgb]{0.38,0.63,0.69}{\textbf{\textit{{#1}}}}}
    \newcommand{\WarningTok}[1]{\textcolor[rgb]{0.38,0.63,0.69}{\textbf{\textit{{#1}}}}}
    
    
    % Define a nice break command that doesn't care if a line doesn't already
    % exist.
    \def\br{\hspace*{\fill} \\* }
    % Math Jax compatability definitions
    \def\gt{>}
    \def\lt{<}
    % Document parameters
    \title{Lab1}
    
    
    

    % Pygments definitions
    
\makeatletter
\def\PY@reset{\let\PY@it=\relax \let\PY@bf=\relax%
    \let\PY@ul=\relax \let\PY@tc=\relax%
    \let\PY@bc=\relax \let\PY@ff=\relax}
\def\PY@tok#1{\csname PY@tok@#1\endcsname}
\def\PY@toks#1+{\ifx\relax#1\empty\else%
    \PY@tok{#1}\expandafter\PY@toks\fi}
\def\PY@do#1{\PY@bc{\PY@tc{\PY@ul{%
    \PY@it{\PY@bf{\PY@ff{#1}}}}}}}
\def\PY#1#2{\PY@reset\PY@toks#1+\relax+\PY@do{#2}}

\expandafter\def\csname PY@tok@w\endcsname{\def\PY@tc##1{\textcolor[rgb]{0.73,0.73,0.73}{##1}}}
\expandafter\def\csname PY@tok@c\endcsname{\let\PY@it=\textit\def\PY@tc##1{\textcolor[rgb]{0.25,0.50,0.50}{##1}}}
\expandafter\def\csname PY@tok@cp\endcsname{\def\PY@tc##1{\textcolor[rgb]{0.74,0.48,0.00}{##1}}}
\expandafter\def\csname PY@tok@k\endcsname{\let\PY@bf=\textbf\def\PY@tc##1{\textcolor[rgb]{0.00,0.50,0.00}{##1}}}
\expandafter\def\csname PY@tok@kp\endcsname{\def\PY@tc##1{\textcolor[rgb]{0.00,0.50,0.00}{##1}}}
\expandafter\def\csname PY@tok@kt\endcsname{\def\PY@tc##1{\textcolor[rgb]{0.69,0.00,0.25}{##1}}}
\expandafter\def\csname PY@tok@o\endcsname{\def\PY@tc##1{\textcolor[rgb]{0.40,0.40,0.40}{##1}}}
\expandafter\def\csname PY@tok@ow\endcsname{\let\PY@bf=\textbf\def\PY@tc##1{\textcolor[rgb]{0.67,0.13,1.00}{##1}}}
\expandafter\def\csname PY@tok@nb\endcsname{\def\PY@tc##1{\textcolor[rgb]{0.00,0.50,0.00}{##1}}}
\expandafter\def\csname PY@tok@nf\endcsname{\def\PY@tc##1{\textcolor[rgb]{0.00,0.00,1.00}{##1}}}
\expandafter\def\csname PY@tok@nc\endcsname{\let\PY@bf=\textbf\def\PY@tc##1{\textcolor[rgb]{0.00,0.00,1.00}{##1}}}
\expandafter\def\csname PY@tok@nn\endcsname{\let\PY@bf=\textbf\def\PY@tc##1{\textcolor[rgb]{0.00,0.00,1.00}{##1}}}
\expandafter\def\csname PY@tok@ne\endcsname{\let\PY@bf=\textbf\def\PY@tc##1{\textcolor[rgb]{0.82,0.25,0.23}{##1}}}
\expandafter\def\csname PY@tok@nv\endcsname{\def\PY@tc##1{\textcolor[rgb]{0.10,0.09,0.49}{##1}}}
\expandafter\def\csname PY@tok@no\endcsname{\def\PY@tc##1{\textcolor[rgb]{0.53,0.00,0.00}{##1}}}
\expandafter\def\csname PY@tok@nl\endcsname{\def\PY@tc##1{\textcolor[rgb]{0.63,0.63,0.00}{##1}}}
\expandafter\def\csname PY@tok@ni\endcsname{\let\PY@bf=\textbf\def\PY@tc##1{\textcolor[rgb]{0.60,0.60,0.60}{##1}}}
\expandafter\def\csname PY@tok@na\endcsname{\def\PY@tc##1{\textcolor[rgb]{0.49,0.56,0.16}{##1}}}
\expandafter\def\csname PY@tok@nt\endcsname{\let\PY@bf=\textbf\def\PY@tc##1{\textcolor[rgb]{0.00,0.50,0.00}{##1}}}
\expandafter\def\csname PY@tok@nd\endcsname{\def\PY@tc##1{\textcolor[rgb]{0.67,0.13,1.00}{##1}}}
\expandafter\def\csname PY@tok@s\endcsname{\def\PY@tc##1{\textcolor[rgb]{0.73,0.13,0.13}{##1}}}
\expandafter\def\csname PY@tok@sd\endcsname{\let\PY@it=\textit\def\PY@tc##1{\textcolor[rgb]{0.73,0.13,0.13}{##1}}}
\expandafter\def\csname PY@tok@si\endcsname{\let\PY@bf=\textbf\def\PY@tc##1{\textcolor[rgb]{0.73,0.40,0.53}{##1}}}
\expandafter\def\csname PY@tok@se\endcsname{\let\PY@bf=\textbf\def\PY@tc##1{\textcolor[rgb]{0.73,0.40,0.13}{##1}}}
\expandafter\def\csname PY@tok@sr\endcsname{\def\PY@tc##1{\textcolor[rgb]{0.73,0.40,0.53}{##1}}}
\expandafter\def\csname PY@tok@ss\endcsname{\def\PY@tc##1{\textcolor[rgb]{0.10,0.09,0.49}{##1}}}
\expandafter\def\csname PY@tok@sx\endcsname{\def\PY@tc##1{\textcolor[rgb]{0.00,0.50,0.00}{##1}}}
\expandafter\def\csname PY@tok@m\endcsname{\def\PY@tc##1{\textcolor[rgb]{0.40,0.40,0.40}{##1}}}
\expandafter\def\csname PY@tok@gh\endcsname{\let\PY@bf=\textbf\def\PY@tc##1{\textcolor[rgb]{0.00,0.00,0.50}{##1}}}
\expandafter\def\csname PY@tok@gu\endcsname{\let\PY@bf=\textbf\def\PY@tc##1{\textcolor[rgb]{0.50,0.00,0.50}{##1}}}
\expandafter\def\csname PY@tok@gd\endcsname{\def\PY@tc##1{\textcolor[rgb]{0.63,0.00,0.00}{##1}}}
\expandafter\def\csname PY@tok@gi\endcsname{\def\PY@tc##1{\textcolor[rgb]{0.00,0.63,0.00}{##1}}}
\expandafter\def\csname PY@tok@gr\endcsname{\def\PY@tc##1{\textcolor[rgb]{1.00,0.00,0.00}{##1}}}
\expandafter\def\csname PY@tok@ge\endcsname{\let\PY@it=\textit}
\expandafter\def\csname PY@tok@gs\endcsname{\let\PY@bf=\textbf}
\expandafter\def\csname PY@tok@gp\endcsname{\let\PY@bf=\textbf\def\PY@tc##1{\textcolor[rgb]{0.00,0.00,0.50}{##1}}}
\expandafter\def\csname PY@tok@go\endcsname{\def\PY@tc##1{\textcolor[rgb]{0.53,0.53,0.53}{##1}}}
\expandafter\def\csname PY@tok@gt\endcsname{\def\PY@tc##1{\textcolor[rgb]{0.00,0.27,0.87}{##1}}}
\expandafter\def\csname PY@tok@err\endcsname{\def\PY@bc##1{\setlength{\fboxsep}{0pt}\fcolorbox[rgb]{1.00,0.00,0.00}{1,1,1}{\strut ##1}}}
\expandafter\def\csname PY@tok@kc\endcsname{\let\PY@bf=\textbf\def\PY@tc##1{\textcolor[rgb]{0.00,0.50,0.00}{##1}}}
\expandafter\def\csname PY@tok@kd\endcsname{\let\PY@bf=\textbf\def\PY@tc##1{\textcolor[rgb]{0.00,0.50,0.00}{##1}}}
\expandafter\def\csname PY@tok@kn\endcsname{\let\PY@bf=\textbf\def\PY@tc##1{\textcolor[rgb]{0.00,0.50,0.00}{##1}}}
\expandafter\def\csname PY@tok@kr\endcsname{\let\PY@bf=\textbf\def\PY@tc##1{\textcolor[rgb]{0.00,0.50,0.00}{##1}}}
\expandafter\def\csname PY@tok@bp\endcsname{\def\PY@tc##1{\textcolor[rgb]{0.00,0.50,0.00}{##1}}}
\expandafter\def\csname PY@tok@fm\endcsname{\def\PY@tc##1{\textcolor[rgb]{0.00,0.00,1.00}{##1}}}
\expandafter\def\csname PY@tok@vc\endcsname{\def\PY@tc##1{\textcolor[rgb]{0.10,0.09,0.49}{##1}}}
\expandafter\def\csname PY@tok@vg\endcsname{\def\PY@tc##1{\textcolor[rgb]{0.10,0.09,0.49}{##1}}}
\expandafter\def\csname PY@tok@vi\endcsname{\def\PY@tc##1{\textcolor[rgb]{0.10,0.09,0.49}{##1}}}
\expandafter\def\csname PY@tok@vm\endcsname{\def\PY@tc##1{\textcolor[rgb]{0.10,0.09,0.49}{##1}}}
\expandafter\def\csname PY@tok@sa\endcsname{\def\PY@tc##1{\textcolor[rgb]{0.73,0.13,0.13}{##1}}}
\expandafter\def\csname PY@tok@sb\endcsname{\def\PY@tc##1{\textcolor[rgb]{0.73,0.13,0.13}{##1}}}
\expandafter\def\csname PY@tok@sc\endcsname{\def\PY@tc##1{\textcolor[rgb]{0.73,0.13,0.13}{##1}}}
\expandafter\def\csname PY@tok@dl\endcsname{\def\PY@tc##1{\textcolor[rgb]{0.73,0.13,0.13}{##1}}}
\expandafter\def\csname PY@tok@s2\endcsname{\def\PY@tc##1{\textcolor[rgb]{0.73,0.13,0.13}{##1}}}
\expandafter\def\csname PY@tok@sh\endcsname{\def\PY@tc##1{\textcolor[rgb]{0.73,0.13,0.13}{##1}}}
\expandafter\def\csname PY@tok@s1\endcsname{\def\PY@tc##1{\textcolor[rgb]{0.73,0.13,0.13}{##1}}}
\expandafter\def\csname PY@tok@mb\endcsname{\def\PY@tc##1{\textcolor[rgb]{0.40,0.40,0.40}{##1}}}
\expandafter\def\csname PY@tok@mf\endcsname{\def\PY@tc##1{\textcolor[rgb]{0.40,0.40,0.40}{##1}}}
\expandafter\def\csname PY@tok@mh\endcsname{\def\PY@tc##1{\textcolor[rgb]{0.40,0.40,0.40}{##1}}}
\expandafter\def\csname PY@tok@mi\endcsname{\def\PY@tc##1{\textcolor[rgb]{0.40,0.40,0.40}{##1}}}
\expandafter\def\csname PY@tok@il\endcsname{\def\PY@tc##1{\textcolor[rgb]{0.40,0.40,0.40}{##1}}}
\expandafter\def\csname PY@tok@mo\endcsname{\def\PY@tc##1{\textcolor[rgb]{0.40,0.40,0.40}{##1}}}
\expandafter\def\csname PY@tok@ch\endcsname{\let\PY@it=\textit\def\PY@tc##1{\textcolor[rgb]{0.25,0.50,0.50}{##1}}}
\expandafter\def\csname PY@tok@cm\endcsname{\let\PY@it=\textit\def\PY@tc##1{\textcolor[rgb]{0.25,0.50,0.50}{##1}}}
\expandafter\def\csname PY@tok@cpf\endcsname{\let\PY@it=\textit\def\PY@tc##1{\textcolor[rgb]{0.25,0.50,0.50}{##1}}}
\expandafter\def\csname PY@tok@c1\endcsname{\let\PY@it=\textit\def\PY@tc##1{\textcolor[rgb]{0.25,0.50,0.50}{##1}}}
\expandafter\def\csname PY@tok@cs\endcsname{\let\PY@it=\textit\def\PY@tc##1{\textcolor[rgb]{0.25,0.50,0.50}{##1}}}

\def\PYZbs{\char`\\}
\def\PYZus{\char`\_}
\def\PYZob{\char`\{}
\def\PYZcb{\char`\}}
\def\PYZca{\char`\^}
\def\PYZam{\char`\&}
\def\PYZlt{\char`\<}
\def\PYZgt{\char`\>}
\def\PYZsh{\char`\#}
\def\PYZpc{\char`\%}
\def\PYZdl{\char`\$}
\def\PYZhy{\char`\-}
\def\PYZsq{\char`\'}
\def\PYZdq{\char`\"}
\def\PYZti{\char`\~}
% for compatibility with earlier versions
\def\PYZat{@}
\def\PYZlb{[}
\def\PYZrb{]}
\makeatother


    % Exact colors from NB
    \definecolor{incolor}{rgb}{0.0, 0.0, 0.5}
    \definecolor{outcolor}{rgb}{0.545, 0.0, 0.0}



    
    % Prevent overflowing lines due to hard-to-break entities
    \sloppy 
    % Setup hyperref package
    \hypersetup{
      breaklinks=true,  % so long urls are correctly broken across lines
      colorlinks=true,
      urlcolor=urlcolor,
      linkcolor=linkcolor,
      citecolor=citecolor,
      }
    % Slightly bigger margins than the latex defaults
    
    \geometry{verbose,tmargin=1in,bmargin=1in,lmargin=1in,rmargin=1in}
    
    

    \begin{document}
    
    
    \maketitle
    
    

    
    \hypertarget{tuux1ea7n-1-uxf4n-tux1eadp-c-con-trux1ecf}{%
\subsection{Tuần 1: Ôn tập C++ (Con
trỏ)}\label{tuux1ea7n-1-uxf4n-tux1eadp-c-con-trux1ecf}}

\hypertarget{giux1ea3ng-viuxean-nguyux1ec5n-ux111ux1ee9c-thux1eafng}{%
\paragraph{Giảng viên: Nguyễn Đức
Thắng}\label{giux1ea3ng-viuxean-nguyux1ec5n-ux111ux1ee9c-thux1eafng}}

\hypertarget{email-thangdnthanglong.edu.vn}{%
\paragraph{Email:
thangdn@thanglong.edu.vn}\label{email-thangdnthanglong.edu.vn}}

\hypertarget{sux1ed1-ux111iux1ec7n-thoux1ea1i-0967953735}{%
\paragraph{Số điện thoại:
0967953735}\label{sux1ed1-ux111iux1ec7n-thoux1ea1i-0967953735}}

    \hypertarget{phux1ea7n-i.-giux1edbi-thiux1ec7u-vux1ec1-kiux1ec3u-dux1eef-liux1ec7u-adt-vector.}{%
\subsubsection{Phần I. Giới thiệu về kiểu dữ liệu ADT
Vector.}\label{phux1ea7n-i.-giux1edbi-thiux1ec7u-vux1ec1-kiux1ec3u-dux1eef-liux1ec7u-adt-vector.}}

Vector có thể hiểu là một mảng có trình tự, giống như với danh sách liên
kết hay một chuỗi thông thường nhưng ``vector'' khác với chuỗi hoăc mảng
thông thường là chúng ta có thể thay đổi kích thước của nó.

    \begin{Verbatim}[commandchars=\\\{\}]
{\color{incolor}In [{\color{incolor}7}]:} \PY{c+cp}{\PYZsh{}}\PY{c+cp}{include} \PY{c+cpf}{\PYZlt{}vector\PYZgt{} // Khai bái thư viện}
        \PY{c+cp}{\PYZsh{}}\PY{c+cp}{include} \PY{c+cpf}{\PYZlt{}iostream\PYZgt{}}
        \PY{k}{using} \PY{k}{namespace} \PY{n}{std}\PY{p}{;}
\end{Verbatim}


    \begin{Verbatim}[commandchars=\\\{\}]
{\color{incolor}In [{\color{incolor}2}]:} \PY{n}{vector}\PY{o}{\PYZlt{}}\PY{k+kt}{double}\PY{o}{\PYZgt{}} \PY{n}{age}\PY{p}{(}\PY{l+m+mi}{4}\PY{p}{)}\PY{p}{;} \PY{c+c1}{// a vector with 4 elements of type double}
        \PY{n}{age}\PY{p}{[}\PY{l+m+mi}{0}\PY{p}{]}\PY{o}{=}\PY{l+m+mf}{0.33}\PY{p}{;}
        \PY{n}{age}\PY{p}{[}\PY{l+m+mi}{1}\PY{p}{]}\PY{o}{=}\PY{l+m+mf}{22.0}\PY{p}{;}
        \PY{n}{age}\PY{p}{[}\PY{l+m+mi}{2}\PY{p}{]}\PY{o}{=}\PY{l+m+mf}{27.2}\PY{p}{;}
        \PY{n}{age}\PY{p}{[}\PY{l+m+mi}{3}\PY{p}{]}\PY{o}{=}\PY{l+m+mf}{54.2}\PY{p}{;}
\end{Verbatim}


    \begin{Verbatim}[commandchars=\\\{\}]
{\color{incolor}In [{\color{incolor}3}]:} \PY{n}{age}
\end{Verbatim}


    \begin{Verbatim}[commandchars=\\\{\}]
(std::vector<double> \&) \{ 0.330000, 22.000000, 27.200000, 54.200000 \}

    \end{Verbatim}

    \begin{figure}
\centering
\includegraphics{attachment:image.png}
\caption{image.png}
\end{figure}

\hypertarget{vux1ec1-cux1a1-bux1ea3n-cuxf3-thux1ec3-huxecnh-dung-class-vector-ux111ux1b0ux1ee3c-thiux1ebft-kux1ebf-nhux1b0-sau}{%
\paragraph{Về cơ bản có thể hình dung class vector được thiết kế như
sau}\label{vux1ec1-cux1a1-bux1ea3n-cuxf3-thux1ec3-huxecnh-dung-class-vector-ux111ux1b0ux1ee3c-thiux1ebft-kux1ebf-nhux1b0-sau}}

    \begin{Shaded}
\begin{Highlighting}[]
\KeywordTok{template}\NormalTok{ <}\KeywordTok{class}\NormalTok{ T>}
\KeywordTok{class}\NormalTok{ vector \{}
    \DataTypeTok{int}\NormalTok{ sz; }\CommentTok{// the size}
\NormalTok{    T* elem; }\CommentTok{// pointer to the first element (of type double)}
\KeywordTok{public}\NormalTok{:}
\NormalTok{    vector(}\DataTypeTok{int}\NormalTok{ s);  }\CommentTok{// constructor: allocate s doubles,}
                    \CommentTok{// let elem point to them}
                    \CommentTok{// store s in sz}
    \DataTypeTok{int}\NormalTok{ size() }\AttributeTok{const}\NormalTok{ \{ }\ControlFlowTok{return}\NormalTok{ sz; \} }\CommentTok{// the current size}
\NormalTok{\};}
\end{Highlighting}
\end{Shaded}

    \hypertarget{tuxednh-linh-ux111ux1ed9ng-cux1ee7a-vector-so-vux1edbi-array}{%
\subsection{Tính linh động của vector so với
array}\label{tuxednh-linh-ux111ux1ed9ng-cux1ee7a-vector-so-vux1edbi-array}}

\hypertarget{khux1edfi-tux1ea1o-linh-ux111ux1ed9ng}{%
\paragraph{1. Khởi tạo linh
động}\label{khux1edfi-tux1ea1o-linh-ux111ux1ed9ng}}

Khởi tạo một vector không cần xác định số phần tử. Mặc định số phần tử
là 0

    \begin{Verbatim}[commandchars=\\\{\}]
{\color{incolor}In [{\color{incolor}4}]:} \PY{n}{vector}\PY{o}{\PYZlt{}}\PY{k+kt}{double}\PY{o}{\PYZgt{}} \PY{n}{second}\PY{p}{;} \PY{c+c1}{//sizeof = 0}
\end{Verbatim}


    \hypertarget{cuxe1c-phux1b0ux1a1ng-thux1ee9c-cux1ee7a-vector}{%
\paragraph{2. Các phương thức của
vector}\label{cuxe1c-phux1b0ux1a1ng-thux1ee9c-cux1ee7a-vector}}

\begin{enumerate}
\def\labelenumi{\arabic{enumi}.}
\tightlist
\item
  \textbf{void push\_back(item)} : Thêm một phần tử vào vector
\item
  \textbf{void pop\_back()} : Xóa một phần tử ở cuối khỏi vector
\item
  \textbf{insert(pos, item)} : Chèn một phần tử vào vector
\item
  \textbf{{[}{]}} : Truy xuất phần tử theo chỉ số
\item
  \textbf{size()} : Trả về số lượng phần tử hiện tại có trong vector
\end{enumerate}

    \begin{Verbatim}[commandchars=\\\{\}]
{\color{incolor}In [{\color{incolor}5}]:} \PY{n}{vector}\PY{o}{\PYZlt{}}\PY{k+kt}{double}\PY{o}{\PYZgt{}} \PY{n}{first}\PY{p}{(}\PY{l+m+mi}{4}\PY{p}{,} \PY{l+m+mi}{50}\PY{p}{)}\PY{p}{;} \PY{c+c1}{// first = \PYZob{} 50, 50, 50, 50\PYZcb{}}
        \PY{n}{first}\PY{p}{.}\PY{n}{push\PYZus{}back} \PY{p}{(}\PY{l+m+mf}{3.1}\PY{p}{)}\PY{p}{;}
        \PY{n}{first}\PY{p}{.}\PY{n}{push\PYZus{}back} \PY{p}{(}\PY{l+m+mf}{2.2}\PY{p}{)}\PY{p}{;}
        \PY{n}{first}\PY{p}{.}\PY{n}{push\PYZus{}back} \PY{p}{(}\PY{l+m+mf}{2.9}\PY{p}{)}\PY{p}{;}
\end{Verbatim}


    \begin{Verbatim}[commandchars=\\\{\}]
{\color{incolor}In [{\color{incolor}6}]:} \PY{n}{first}
\end{Verbatim}


    \begin{Verbatim}[commandchars=\\\{\}]
(std::vector<double> \&) \{ 50.000000, 50.000000, 50.000000, 50.000000, 3.100000, 2.200000, 2.900000 \}

    \end{Verbatim}

    \begin{Verbatim}[commandchars=\\\{\}]
{\color{incolor}In [{\color{incolor}7}]:} \PY{n}{first}\PY{p}{[}\PY{l+m+mi}{2}\PY{p}{]}
\end{Verbatim}


    \begin{Verbatim}[commandchars=\\\{\}]
(double) 50.000000

    \end{Verbatim}

    \begin{Verbatim}[commandchars=\\\{\}]
{\color{incolor}In [{\color{incolor}8}]:} \PY{n}{first}\PY{p}{.}\PY{n}{pop\PYZus{}back}\PY{p}{(}\PY{p}{)}\PY{p}{;}
        \PY{n}{first}
\end{Verbatim}


    \begin{Verbatim}[commandchars=\\\{\}]
(std::vector<double> \&) \{ 50.000000, 50.000000, 50.000000, 50.000000, 3.100000, 2.200000 \}

    \end{Verbatim}

    \begin{Verbatim}[commandchars=\\\{\}]
{\color{incolor}In [{\color{incolor}9}]:} \PY{n}{cout} \PY{o}{\PYZlt{}}\PY{o}{\PYZlt{}} \PY{n}{first}\PY{p}{.}\PY{n}{at}\PY{p}{(}\PY{l+m+mi}{5}\PY{p}{)} \PY{o}{\PYZlt{}}\PY{o}{\PYZlt{}} \PY{n}{endl} \PY{o}{\PYZlt{}}\PY{o}{\PYZlt{}} \PY{n}{first}\PY{p}{[}\PY{l+m+mi}{6}\PY{p}{]}\PY{p}{;}
\end{Verbatim}


    \begin{Verbatim}[commandchars=\\\{\}]
2.2
2.9
    \end{Verbatim}

    Với phương thức \textbf{swap()} cho phép hoán đổi nội dung giữa 2
vector, để hiểu rõ chức năng của nó xem ví dụ sau đây:

    \begin{Verbatim}[commandchars=\\\{\}]
{\color{incolor}In [{\color{incolor}10}]:} \PY{n}{vector}\PY{o}{\PYZlt{}}\PY{k+kt}{int}\PY{o}{\PYZgt{}} \PY{n}{foo} \PY{p}{(}\PY{l+m+mi}{3}\PY{p}{,}\PY{l+m+mi}{100}\PY{p}{)}\PY{p}{;}   \PY{c+c1}{// three ints with a value of 100}
         \PY{n}{vector}\PY{o}{\PYZlt{}}\PY{k+kt}{int}\PY{o}{\PYZgt{}} \PY{n}{bar} \PY{p}{(}\PY{l+m+mi}{5}\PY{p}{,}\PY{l+m+mi}{200}\PY{p}{)}\PY{p}{;}   \PY{c+c1}{// five ints with a value of 200}
\end{Verbatim}


    \begin{Verbatim}[commandchars=\\\{\}]
{\color{incolor}In [{\color{incolor}11}]:} \PY{n}{foo}
\end{Verbatim}


    \begin{Verbatim}[commandchars=\\\{\}]
(std::vector<int> \&) \{ 100, 100, 100 \}

    \end{Verbatim}

    \begin{Verbatim}[commandchars=\\\{\}]
{\color{incolor}In [{\color{incolor}12}]:} \PY{n}{bar}
\end{Verbatim}


    \begin{Verbatim}[commandchars=\\\{\}]
(std::vector<int> \&) \{ 200, 200, 200, 200, 200 \}

    \end{Verbatim}

    \begin{Verbatim}[commandchars=\\\{\}]
{\color{incolor}In [{\color{incolor}13}]:} \PY{n}{foo}\PY{p}{.}\PY{n}{swap}\PY{p}{(}\PY{n}{bar}\PY{p}{)}\PY{p}{;}
\end{Verbatim}


    \begin{Verbatim}[commandchars=\\\{\}]
{\color{incolor}In [{\color{incolor}14}]:} \PY{n}{foo}
\end{Verbatim}


    \begin{Verbatim}[commandchars=\\\{\}]
(std::vector<int> \&) \{ 200, 200, 200, 200, 200 \}

    \end{Verbatim}

    \begin{Verbatim}[commandchars=\\\{\}]
{\color{incolor}In [{\color{incolor}15}]:} \PY{n}{bar}
\end{Verbatim}


    \begin{Verbatim}[commandchars=\\\{\}]
(std::vector<int> \&) \{ 100, 100, 100 \}

    \end{Verbatim}

    Với phương thức \textbf{empty()} cho phép kiểm tra ``vector'' có rỗng
hay không, nếu có sẽ trả về true và ngược lại:

    \begin{Verbatim}[commandchars=\\\{\}]
{\color{incolor}In [{\color{incolor}16}]:} \PY{k}{if}\PY{p}{(}\PY{n}{second}\PY{p}{.}\PY{n}{empty}\PY{p}{(}\PY{p}{)} \PY{o}{=}\PY{o}{=} \PY{n+nb}{true}\PY{p}{)}
         \PY{p}{\PYZob{}}
                \PY{n}{cout} \PY{o}{\PYZlt{}}\PY{o}{\PYZlt{}} \PY{l+s}{\PYZdq{}}\PY{l+s}{No values in second }\PY{l+s+se}{\PYZbs{}n}\PY{l+s}{\PYZdq{}}\PY{p}{;}
         \PY{p}{\PYZcb{}}
\end{Verbatim}


    \begin{Verbatim}[commandchars=\\\{\}]
No values in second 

    \end{Verbatim}

    \begin{Verbatim}[commandchars=\\\{\}]
{\color{incolor}In [{\color{incolor}17}]:} \PY{n}{cout}\PY{o}{\PYZlt{}}\PY{o}{\PYZlt{}}\PY{n}{first}\PY{p}{.}\PY{n}{size}\PY{p}{(}\PY{p}{)}\PY{p}{;}
\end{Verbatim}


    \begin{Verbatim}[commandchars=\\\{\}]
6
    \end{Verbatim}

    Dùng hàm \textbf{clear()} để xóa hết các phần tử của ``vector'':

    \begin{Verbatim}[commandchars=\\\{\}]
{\color{incolor}In [{\color{incolor}18}]:} \PY{n}{first}\PY{p}{.}\PY{n}{push\PYZus{}back} \PY{p}{(}\PY{l+m+mf}{3.1}\PY{p}{)}\PY{p}{;}
         \PY{n}{first}\PY{p}{.}\PY{n}{push\PYZus{}back} \PY{p}{(}\PY{l+m+mf}{2.2}\PY{p}{)}\PY{p}{;}
         \PY{n}{first}\PY{p}{.}\PY{n}{push\PYZus{}back} \PY{p}{(}\PY{l+m+mf}{2.9}\PY{p}{)}\PY{p}{;}
          
         \PY{n}{first}\PY{p}{.}\PY{n}{clear}\PY{p}{(}\PY{p}{)}\PY{p}{;}
\end{Verbatim}


    \hypertarget{phux1ea7n-2-memory-addresses-and-pointers}{%
\subsubsection{Phần 2: Memory, addresses, and
pointers}\label{phux1ea7n-2-memory-addresses-and-pointers}}

Khi chúng ta khởi tạo một biến tức đồng nghĩa chúng ta đặt giá trị vào
một địa chỉ ô nhớ. \includegraphics{attachment:image.png}

    \begin{Verbatim}[commandchars=\\\{\}]
{\color{incolor}In [{\color{incolor}19}]:} \PY{k+kt}{int} \PY{n}{var} \PY{o}{=} \PY{l+m+mi}{17}\PY{p}{;}
\end{Verbatim}


    Giả sử chúng ta muốn lưu trữ địa chỉ ô nhớ của biến var. Chúng ta có
khái niệm \textbf{con trỏ}.

    \begin{Verbatim}[commandchars=\\\{\}]
{\color{incolor}In [{\color{incolor}20}]:} \PY{k+kt}{int}\PY{o}{*} \PY{n}{ptr} \PY{o}{=} \PY{o}{\PYZam{}}\PY{n}{var}\PY{p}{;}\PY{c+c1}{// ptr holds the address of var}
\end{Verbatim}


    Toán tử \textbf{\&} được sử dụng để lấy địa chỉ của một đối tượng. Như
vậy biến \textbf{var} là địa chỉ ô nhớ là

    \begin{Verbatim}[commandchars=\\\{\}]
{\color{incolor}In [{\color{incolor}21}]:} \PY{n}{ptr}
\end{Verbatim}


    \begin{Verbatim}[commandchars=\\\{\}]
(int *) 0x7ff1bc9bc138

    \end{Verbatim}

    Giả sử địa chỉ của \textbf{var} nhận được là 0x4096. Như vậy ptr sẽ có
giá trị tương ứng là 0x4096 \textasciitilde{} \(2^{12}\).
\includegraphics{attachment:image.png}

    \begin{Verbatim}[commandchars=\\\{\}]
{\color{incolor}In [{\color{incolor}22}]:} \PY{k+kt}{int} \PY{n}{x} \PY{o}{=} \PY{l+m+mi}{17}\PY{p}{;}
         \PY{k+kt}{int}\PY{o}{*} \PY{n}{pii} \PY{o}{=} \PY{o}{\PYZam{}}\PY{n}{x}\PY{p}{;} \PY{c+c1}{// pointer to int}
         \PY{k+kt}{double} \PY{n}{e} \PY{o}{=} \PY{l+m+mf}{2.71828}\PY{p}{;}
         \PY{k+kt}{double}\PY{o}{*} \PY{n}{pdi} \PY{o}{=} \PY{o}{\PYZam{}}\PY{n}{e}\PY{p}{;} \PY{c+c1}{// pointer to double}
\end{Verbatim}


    \begin{Verbatim}[commandchars=\\\{\}]
{\color{incolor}In [{\color{incolor}23}]:} \PY{n}{pdi}
\end{Verbatim}


    \begin{Verbatim}[commandchars=\\\{\}]
(double *) 0x7ff1bc9bc160

    \end{Verbatim}

    Nếu chúng ta muốn lấy giá trị trực tiếp thông qua địa chỉ ta chỉ cần sử
dụng toán tử \textbf{*}

    \begin{Verbatim}[commandchars=\\\{\}]
{\color{incolor}In [{\color{incolor}24}]:} \PY{o}{*}\PY{n}{pdi}
\end{Verbatim}


    \begin{Verbatim}[commandchars=\\\{\}]
(double) 2.718280

    \end{Verbatim}

    \begin{Verbatim}[commandchars=\\\{\}]
{\color{incolor}In [{\color{incolor}25}]:} \PY{o}{*}\PY{n}{pii}
\end{Verbatim}


    \begin{Verbatim}[commandchars=\\\{\}]
(int) 17

    \end{Verbatim}

    \begin{Verbatim}[commandchars=\\\{\}]
{\color{incolor}In [{\color{incolor}26}]:} \PY{k+kt}{char} \PY{n}{ch1} \PY{o}{=} \PY{l+s+sc}{\PYZsq{}}\PY{l+s+sc}{a}\PY{l+s+sc}{\PYZsq{}}\PY{p}{;}
         \PY{k+kt}{char} \PY{n}{ch2} \PY{o}{=} \PY{l+s+sc}{\PYZsq{}}\PY{l+s+sc}{b}\PY{l+s+sc}{\PYZsq{}}\PY{p}{;}
         \PY{k+kt}{char} \PY{n}{ch3} \PY{o}{=} \PY{l+s+sc}{\PYZsq{}}\PY{l+s+sc}{c}\PY{l+s+sc}{\PYZsq{}}\PY{p}{;}
         \PY{k+kt}{char} \PY{n}{ch4} \PY{o}{=} \PY{l+s+sc}{\PYZsq{}}\PY{l+s+sc}{d}\PY{l+s+sc}{\PYZsq{}}\PY{p}{;}
         \PY{k+kt}{char}\PY{o}{*} \PY{n}{pix} \PY{o}{=} \PY{o}{\PYZam{}}\PY{n}{ch3}\PY{p}{;}
\end{Verbatim}


    \begin{Verbatim}[commandchars=\\\{\}]
{\color{incolor}In [{\color{incolor}27}]:} \PY{o}{*}\PY{n}{pix}
\end{Verbatim}


    \begin{Verbatim}[commandchars=\\\{\}]
(char) 'c'

    \end{Verbatim}

    \begin{figure}
\centering
\includegraphics{attachment:image.png}
\caption{image.png}
\end{figure}

    \hypertarget{free-store-and-pointers---cux1ea5p-phuxe1t-ux111ux1ed9ng}{%
\paragraph{Free store and pointers - Cấp phát
động}\label{free-store-and-pointers---cux1ea5p-phuxe1t-ux111ux1ed9ng}}

Một sự hiểu biết sâu về cách bộ nhớ động thực sự làm việc trong C/C++ là
cốt yếu để trở thành một lập trình viên C/C++ giỏi. Bộ nhớ trong chương
trình C/C++ của bạn được phân thành hai phần:

\begin{itemize}
\item
  \textbf{Stack}: Tất cả biến được khai báo bên trong hàm sẽ nhận bộ nhớ
  từ stack trong C/C++.
\item
  \textbf{Free store (Heap)}: Được sử dụng để cấp phát bộ nhớ động khi
  chương trình chạy. \includegraphics{attachment:image.png}
\end{itemize}

Thường thì ta không biết trước bao nhiêu bộ nhớ bạn sẽ cần để lưu thông
tin cụ thể trong một biến đã được định nghĩa và kích cỡ bộ nhớ cần thiết
có thể được quyết định tại run time.

Bạn có thể cấp phát bộ nhớ tại run time bên trong Heap cho biến đó với
một kiểu đã cho bởi sử dụng một toán tử đặc biệt trong C/C++ mà trả về
địa chỉ của không gian đã cấp phát. Toán tử này gọi là toán tử
\textbf{new} trong C/C++.

Nếu bạn không cần thiết bộ nhớ động đã cấp phát nữa, ta có thể sử dụng
toán tử \textbf{delete} trong C/C++, sẽ giải phóng bộ nhớ đã được cấp
phát trước đó bởi toán tử \textbf{new}.

    \begin{Verbatim}[commandchars=\\\{\}]
{\color{incolor}In [{\color{incolor}28}]:} \PY{c+c1}{// Cấp phát vùng nhớ}
         \PY{k+kt}{double} \PY{o}{*}\PY{n}{arr} \PY{o}{=} \PY{k}{new} \PY{k+kt}{double}\PY{p}{;} \PY{c+c1}{// Cấp phát con trỏ int a}
         \PY{k+kt}{int} \PY{n}{n} \PY{o}{=} \PY{l+m+mi}{10}\PY{p}{;}
         \PY{k+kt}{int} \PY{o}{*}\PY{n}{p} \PY{o}{=} \PY{k}{new} \PY{k+kt}{int}\PY{p}{[}\PY{n}{n}\PY{p}{]}\PY{p}{;} \PY{c+c1}{// Cấp phát mảng int arr n phần tử}
          
         \PY{c+c1}{// Xoá vùng nhớ}
         \PY{k}{delete} \PY{n}{arr}\PY{p}{;} \PY{c+c1}{// Xoá con trỏ}
         \PY{k}{delete}\PY{p}{[}\PY{p}{]} \PY{n}{p}\PY{p}{;} \PY{c+c1}{// Xoá mảng động}
\end{Verbatim}


    \begin{figure}
\centering
\includegraphics{attachment:image.png}
\caption{image.png}
\end{figure}

    \begin{Verbatim}[commandchars=\\\{\}]
{\color{incolor}In [{\color{incolor}29}]:} \PY{k+kt}{int}\PY{o}{*} \PY{n}{pi} \PY{o}{=} \PY{k}{new} \PY{k+kt}{int}\PY{p}{;}
         \PY{k+kt}{int}\PY{o}{*} \PY{n}{qi} \PY{o}{=} \PY{k}{new} \PY{k+kt}{int}\PY{p}{[}\PY{l+m+mi}{4}\PY{p}{]}\PY{p}{;} \PY{c+c1}{// allocate one int}
         \PY{c+c1}{// allocate 4 ints (an array of 4 ints)}
         \PY{k+kt}{double}\PY{o}{*} \PY{n}{pd} \PY{o}{=} \PY{k}{new} \PY{k+kt}{double}\PY{p}{;}
         \PY{k+kt}{double}\PY{o}{*} \PY{n}{qd} \PY{o}{=} \PY{k}{new} \PY{k+kt}{double}\PY{p}{[}\PY{n}{n}\PY{p}{]}\PY{p}{;} \PY{c+c1}{// allocate one double}
         \PY{c+c1}{// allocate n doubles (an array of n doubles)}
\end{Verbatim}


    \begin{figure}
\centering
\includegraphics{attachment:image.png}
\caption{image.png}
\end{figure}

    \hypertarget{truy-xuux1ea5t-cuxe1c-phux1ea7n-tux1eed-trong-mux1ea3ng-ux111ux1ed9ng}{%
\paragraph{Truy xuất các phần tử trong mảng
động}\label{truy-xuux1ea5t-cuxe1c-phux1ea7n-tux1eed-trong-mux1ea3ng-ux111ux1ed9ng}}

    \begin{Verbatim}[commandchars=\\\{\}]
{\color{incolor}In [{\color{incolor}30}]:} \PY{k+kt}{double}\PY{o}{*} \PY{n}{pc} \PY{o}{=} \PY{k}{new} \PY{k+kt}{double}\PY{p}{[}\PY{l+m+mi}{4}\PY{p}{]}\PY{p}{;}\PY{c+c1}{// allocate 4 doubles on the free store}
         \PY{k+kt}{double} \PY{n}{xc} \PY{o}{=} \PY{o}{*}\PY{n}{pc}\PY{p}{;} \PY{c+c1}{// read the (first) object pointed to by p}
         \PY{k+kt}{double} \PY{n}{yc} \PY{o}{=} \PY{n}{pc}\PY{p}{[}\PY{l+m+mi}{2}\PY{p}{]}\PY{p}{;} \PY{c+c1}{// read the 3rd object pointed to by p}
\end{Verbatim}


    \begin{Verbatim}[commandchars=\\\{\}]
{\color{incolor}In [{\color{incolor}31}]:} \PY{n}{xc}
\end{Verbatim}


    \begin{Verbatim}[commandchars=\\\{\}]
(double) 0.000000

    \end{Verbatim}

    \begin{Verbatim}[commandchars=\\\{\}]
{\color{incolor}In [{\color{incolor}32}]:} \PY{n}{yc}
\end{Verbatim}


    \begin{Verbatim}[commandchars=\\\{\}]
(double) 62183289155097584584720350502526582656388213139763713719448292850826136845857531898307640988239856300874899184532375406353310359351114389212045850250235707716316298890002300928.000000

    \end{Verbatim}

    \begin{Verbatim}[commandchars=\\\{\}]
{\color{incolor}In [{\color{incolor}33}]:} \PY{o}{*}\PY{n}{pc} \PY{o}{=} \PY{l+m+mf}{7.7}\PY{p}{;} \PY{c+c1}{// write to the (first) object pointed to by p}
         \PY{n}{pc}\PY{p}{[}\PY{l+m+mi}{2}\PY{p}{]} \PY{o}{=} \PY{l+m+mf}{9.9}\PY{p}{;} \PY{c+c1}{// write to the 3rd object pointed to by p}
\end{Verbatim}


    \begin{Verbatim}[commandchars=\\\{\}]
{\color{incolor}In [{\color{incolor}34}]:} \PY{k}{for}\PY{p}{(}\PY{k+kt}{int} \PY{n}{i} \PY{o}{=} \PY{l+m+mi}{0}\PY{p}{;}\PY{n}{i}\PY{o}{\PYZlt{}} \PY{l+m+mi}{4}\PY{p}{;}\PY{n}{i}\PY{o}{+}\PY{o}{+}\PY{p}{)}
             \PY{n}{cout}\PY{o}{\PYZlt{}}\PY{o}{\PYZlt{}}\PY{n}{pc}\PY{p}{[}\PY{n}{i}\PY{p}{]}\PY{o}{\PYZlt{}}\PY{o}{\PYZlt{}}\PY{l+s}{\PYZdq{}}\PY{l+s}{ }\PY{l+s}{\PYZdq{}}\PY{p}{;}
\end{Verbatim}


    \begin{Verbatim}[commandchars=\\\{\}]
7.7 4.72298e+170 9.9 1.3493e+241 
    \end{Verbatim}

    \begin{Verbatim}[commandchars=\\\{\}]
{\color{incolor}In [{\color{incolor}35}]:} \PY{o}{*}\PY{n}{pc}
\end{Verbatim}


    \begin{Verbatim}[commandchars=\\\{\}]
(double) 7.700000

    \end{Verbatim}

    \begin{Verbatim}[commandchars=\\\{\}]
{\color{incolor}In [{\color{incolor}36}]:} \PY{o}{*}\PY{p}{(}\PY{n}{pc}\PY{o}{+}\PY{l+m+mi}{2}\PY{p}{)}
\end{Verbatim}


    \begin{Verbatim}[commandchars=\\\{\}]
(double) 9.900000

    \end{Verbatim}

    \begin{Verbatim}[commandchars=\\\{\}]
{\color{incolor}In [{\color{incolor}37}]:} \PY{k+kt}{double}\PY{o}{*} \PY{n}{pz} \PY{o}{=} \PY{k}{new} \PY{k+kt}{double}\PY{p}{;} \PY{c+c1}{// allocate a double}
         \PY{k+kt}{double}\PY{o}{*} \PY{n}{qz} \PY{o}{=} \PY{k}{new} \PY{k+kt}{double}\PY{p}{[}\PY{l+m+mi}{1000}\PY{p}{]}\PY{p}{;} \PY{c+c1}{// allocate 1000 doubles}
         
         \PY{n}{qz}\PY{p}{[}\PY{l+m+mi}{700}\PY{p}{]} \PY{o}{=} \PY{l+m+mf}{7.7}\PY{p}{;}\PY{c+c1}{// fine}
\end{Verbatim}


    \begin{Verbatim}[commandchars=\\\{\}]
{\color{incolor}In [{\color{incolor}38}]:} \PY{n}{qz} \PY{o}{=} \PY{n}{pz}\PY{p}{;} \PY{c+c1}{// let q point to the same as p}
\end{Verbatim}


    \begin{Verbatim}[commandchars=\\\{\}]
{\color{incolor}In [{\color{incolor}39}]:} \PY{o}{*}\PY{n}{qz}
\end{Verbatim}


    \begin{Verbatim}[commandchars=\\\{\}]
(double) 0.000000

    \end{Verbatim}

    \begin{Verbatim}[commandchars=\\\{\}]
{\color{incolor}In [{\color{incolor}40}]:} \PY{n}{cout}\PY{o}{\PYZlt{}}\PY{o}{\PYZlt{}}\PY{n}{qz}\PY{p}{[}\PY{l+m+mi}{700}\PY{p}{]}\PY{p}{;} \PY{c+c1}{// out\PYZhy{}of\PYZhy{}range access!}
\end{Verbatim}


    \begin{Verbatim}[commandchars=\\\{\}]
2.41156e-316
    \end{Verbatim}

    \begin{figure}
\centering
\includegraphics{attachment:image.png}
\caption{image.png}
\end{figure}

    \hypertarget{khux1edfi-tux1ea1o-cux1ea5p-phuxe1t-ux111ux1ed9ng}{%
\paragraph{Khởi tạo cấp phát
động}\label{khux1edfi-tux1ea1o-cux1ea5p-phuxe1t-ux111ux1ed9ng}}

    \begin{Verbatim}[commandchars=\\\{\}]
{\color{incolor}In [{\color{incolor}40}]:} \PY{k+kt}{double}\PY{o}{*} \PY{n}{p0}\PY{p}{;}
         \PY{k+kt}{double}\PY{o}{*} \PY{n}{p1} \PY{o}{=} \PY{k}{new} \PY{k+kt}{double}\PY{p}{;}
         \PY{k+kt}{double}\PY{o}{*} \PY{n}{p2} \PY{o}{=} \PY{k}{new} \PY{k+kt}{double}\PY{p}{\PYZob{}}\PY{l+m+mf}{5.5}\PY{p}{\PYZcb{}}\PY{p}{;}
         \PY{k+kt}{double}\PY{o}{*} \PY{n}{p3} \PY{o}{=} \PY{k}{new} \PY{k+kt}{double}\PY{p}{[}\PY{l+m+mi}{5}\PY{p}{]}\PY{p}{;}
\end{Verbatim}


    \begin{Verbatim}[commandchars=\\\{\}]
{\color{incolor}In [{\color{incolor}41}]:} \PY{k+kt}{double}\PY{o}{*} \PY{n}{p4} \PY{o}{=} \PY{k}{new} \PY{k+kt}{double}\PY{p}{[}\PY{l+m+mi}{5}\PY{p}{]} \PY{p}{\PYZob{}}\PY{l+m+mi}{0}\PY{p}{,}\PY{l+m+mi}{1}\PY{p}{,}\PY{l+m+mi}{2}\PY{p}{,}\PY{l+m+mi}{3}\PY{p}{,}\PY{l+m+mi}{4}\PY{p}{\PYZcb{}}\PY{p}{;}
\end{Verbatim}


    \hypertarget{the-null-pointer}{%
\subsubsection{The null pointer}\label{the-null-pointer}}

    \begin{Verbatim}[commandchars=\\\{\}]
{\color{incolor}In [{\color{incolor}42}]:} \PY{k+kt}{double} \PY{o}{*}\PY{n}{p5} \PY{o}{=} \PY{k}{nullptr}\PY{p}{;}\PY{c+c1}{// the null pointer}
\end{Verbatim}


    \begin{Verbatim}[commandchars=\\\{\}]
{\color{incolor}In [{\color{incolor}43}]:} \PY{k}{if}\PY{p}{(}\PY{n}{p5} \PY{o}{=}\PY{o}{=} \PY{k}{nullptr}\PY{p}{)}
             \PY{n}{cout}\PY{o}{\PYZlt{}}\PY{o}{\PYZlt{}}\PY{l+s}{\PYZdq{}}\PY{l+s}{null pointer}\PY{l+s}{\PYZdq{}}\PY{p}{;}
         \PY{k}{else}
             \PY{n}{cout}\PY{o}{\PYZlt{}}\PY{o}{\PYZlt{}}\PY{l+s}{\PYZdq{}}\PY{l+s}{not null pointer}\PY{l+s}{\PYZdq{}}\PY{p}{;}
\end{Verbatim}


    \begin{Verbatim}[commandchars=\\\{\}]
null pointer
    \end{Verbatim}

    \begin{Verbatim}[commandchars=\\\{\}]
{\color{incolor}In [{\color{incolor}44}]:} \PY{o}{*}\PY{n}{p5}
\end{Verbatim}


    \begin{Verbatim}[commandchars=\\\{\}]
input\_line\_71:2:3: warning: null passed to a callee that requires a non-null argument [-Wnonnull]
 *p5
  \^{}\textasciitilde{}

    \end{Verbatim}

    \begin{Verbatim}[commandchars=\\\{\}]
{\color{incolor}In [{\color{incolor}45}]:} \PY{k}{delete} \PY{n}{p5}\PY{p}{;}
\end{Verbatim}


    \hypertarget{xuxe2y-dux1ef1ng-class-vector}{%
\subsubsection{Xây dựng class
vector}\label{xuxe2y-dux1ef1ng-class-vector}}

    \begin{Shaded}
\begin{Highlighting}[]
\KeywordTok{class}\NormalTok{ vector \{}
    \DataTypeTok{int}\NormalTok{ sz;}
    \DataTypeTok{double}\NormalTok{* elem;}
\KeywordTok{public}\NormalTok{:}
\NormalTok{    vector(}\DataTypeTok{int}\NormalTok{ s):sz\{s\},elem\{}\KeywordTok{new} \DataTypeTok{double}\NormalTok{[s]\}}
\NormalTok{    \{}
        \ControlFlowTok{for}\NormalTok{ (}\DataTypeTok{int}\NormalTok{ i=}\DecValTok{0}\NormalTok{; i<s; ++i) elem[i]=}\DecValTok{0}\NormalTok{;}
\NormalTok{    \}}
    
    \DataTypeTok{int}\NormalTok{ size() }\AttributeTok{const}\NormalTok{ \{ }\ControlFlowTok{return}\NormalTok{ sz; \}}
    \CommentTok{// . . .}
    \DataTypeTok{double}\NormalTok{ get(}\DataTypeTok{int}\NormalTok{ n) }\AttributeTok{const} 
\NormalTok{    \{ }
        \ControlFlowTok{return}\NormalTok{ elem[n]; }
\NormalTok{    \}}
    \DataTypeTok{void}\NormalTok{ set(}\DataTypeTok{int}\NormalTok{ n, }\DataTypeTok{double}\NormalTok{ v) }
\NormalTok{    \{ }
\NormalTok{        elem[n]=v; }
\NormalTok{    \}}
\NormalTok{    ~vector()}
\NormalTok{    \{ }
        \KeywordTok{delete}\NormalTok{[] elem; }
\NormalTok{    \}}
\NormalTok{\};}
\end{Highlighting}
\end{Shaded}

    \begin{Verbatim}[commandchars=\\\{\}]
{\color{incolor}In [{\color{incolor}1}]:} \PY{k}{class} \PY{n+nc}{vectorCF212} \PY{p}{\PYZob{}}
            \PY{k+kt}{int} \PY{n}{sz}\PY{p}{;}
            \PY{k+kt}{double}\PY{o}{*} \PY{n}{elem}\PY{p}{;}
        \PY{k}{public}\PY{o}{:}
            \PY{n}{vectorCF212}\PY{p}{(}\PY{k+kt}{int} \PY{n}{s}\PY{p}{)}\PY{o}{:}\PY{n}{sz}\PY{p}{\PYZob{}}\PY{n}{s}\PY{p}{\PYZcb{}}\PY{p}{,}\PY{n}{elem}\PY{p}{\PYZob{}}\PY{k}{new} \PY{k+kt}{double}\PY{p}{[}\PY{n}{s}\PY{p}{]}\PY{p}{\PYZcb{}}
            \PY{p}{\PYZob{}}
                \PY{k}{for} \PY{p}{(}\PY{k+kt}{int} \PY{n}{i}\PY{o}{=}\PY{l+m+mi}{0}\PY{p}{;} \PY{n}{i}\PY{o}{\PYZlt{}}\PY{n}{s}\PY{p}{;} \PY{o}{+}\PY{o}{+}\PY{n}{i}\PY{p}{)} \PY{n}{elem}\PY{p}{[}\PY{n}{i}\PY{p}{]}\PY{o}{=}\PY{l+m+mi}{0}\PY{p}{;}
            \PY{p}{\PYZcb{}}
            
            \PY{k+kt}{int} \PY{n}{size}\PY{p}{(}\PY{p}{)} \PY{k}{const} \PY{p}{\PYZob{}} \PY{k}{return} \PY{n}{sz}\PY{p}{;} \PY{p}{\PYZcb{}}
            \PY{c+c1}{// . . .}
            \PY{k+kt}{double} \PY{n}{get}\PY{p}{(}\PY{k+kt}{int} \PY{n}{n}\PY{p}{)} \PY{k}{const} 
            \PY{p}{\PYZob{}} 
                \PY{k}{return} \PY{n}{elem}\PY{p}{[}\PY{n}{n}\PY{p}{]}\PY{p}{;} 
            \PY{p}{\PYZcb{}}
            \PY{k+kt}{void} \PY{n}{set}\PY{p}{(}\PY{k+kt}{int} \PY{n}{n}\PY{p}{,} \PY{k+kt}{double} \PY{n}{v}\PY{p}{)} 
            \PY{p}{\PYZob{}} 
                \PY{n}{elem}\PY{p}{[}\PY{n}{n}\PY{p}{]}\PY{o}{=}\PY{n}{v}\PY{p}{;} 
            \PY{p}{\PYZcb{}}
            \PY{o}{\PYZti{}}\PY{n}{vectorCF212}\PY{p}{(}\PY{p}{)}
            \PY{p}{\PYZob{}} 
                \PY{k}{delete}\PY{p}{[}\PY{p}{]} \PY{n}{elem}\PY{p}{;} 
            \PY{p}{\PYZcb{}}
        \PY{p}{\PYZcb{}}\PY{p}{;}
\end{Verbatim}


    \begin{Verbatim}[commandchars=\\\{\}]
{\color{incolor}In [{\color{incolor}3}]:} \PY{n}{vectorCF212} \PY{n+nf}{newVar}\PY{p}{(}\PY{l+m+mi}{10}\PY{p}{)}\PY{p}{;}
\end{Verbatim}


    \begin{Verbatim}[commandchars=\\\{\}]
{\color{incolor}In [{\color{incolor}4}]:} \PY{n}{newVar}\PY{p}{.}\PY{n}{size}\PY{p}{(}\PY{p}{)}
\end{Verbatim}


    \begin{Verbatim}[commandchars=\\\{\}]
(int) 10

    \end{Verbatim}

    \begin{Verbatim}[commandchars=\\\{\}]
{\color{incolor}In [{\color{incolor}5}]:} \PY{n}{newVar}\PY{p}{.}\PY{n}{set}\PY{p}{(}\PY{l+m+mi}{5}\PY{p}{,}\PY{l+m+mi}{10}\PY{p}{)}\PY{p}{;}
        \PY{n}{newVar}\PY{p}{.}\PY{n}{get}\PY{p}{(}\PY{l+m+mi}{5}\PY{p}{)}
\end{Verbatim}


    \begin{Verbatim}[commandchars=\\\{\}]
(double) 10.000000

    \end{Verbatim}

    \begin{Verbatim}[commandchars=\\\{\}]
{\color{incolor}In [{\color{incolor}8}]:} \PY{k}{for}\PY{p}{(}\PY{k+kt}{int} \PY{n}{i} \PY{o}{=} \PY{l+m+mi}{0}\PY{p}{;} \PY{n}{i}\PY{o}{\PYZlt{}} \PY{n}{newVar}\PY{p}{.}\PY{n}{size}\PY{p}{(}\PY{p}{)}\PY{p}{;}\PY{n}{i}\PY{o}{+}\PY{o}{+}\PY{p}{)}
            \PY{n}{cout}\PY{o}{\PYZlt{}}\PY{o}{\PYZlt{}}\PY{n}{newVar}\PY{p}{.}\PY{n}{get}\PY{p}{(}\PY{n}{i}\PY{p}{)}\PY{o}{\PYZlt{}}\PY{o}{\PYZlt{}}\PY{l+s}{\PYZdq{}}\PY{l+s}{ }\PY{l+s}{\PYZdq{}}\PY{p}{;}
\end{Verbatim}


    \begin{Verbatim}[commandchars=\\\{\}]
0 0 0 0 0 10 0 0 0 0 
    \end{Verbatim}

    \hypertarget{pointers-to-class-objects}{%
\subsubsection{Pointers to class
objects}\label{pointers-to-class-objects}}

Ta cần dùng toán tử truy xuất \textbf{-\textgreater{}}

    \begin{Verbatim}[commandchars=\\\{\}]
{\color{incolor}In [{\color{incolor}9}]:} \PY{n}{vectorCF212} \PY{o}{*}\PY{n}{newObjectPtr} \PY{o}{=} \PY{k}{new} \PY{n}{vectorCF212}\PY{p}{(}\PY{l+m+mi}{10}\PY{p}{)}\PY{p}{;}
\end{Verbatim}


    \begin{Verbatim}[commandchars=\\\{\}]
{\color{incolor}In [{\color{incolor}10}]:} \PY{n}{newObjectPtr}\PY{o}{\PYZhy{}}\PY{o}{\PYZgt{}}\PY{n}{size}\PY{p}{(}\PY{p}{)}
\end{Verbatim}


    \begin{Verbatim}[commandchars=\\\{\}]
(int) 10

    \end{Verbatim}

    \begin{Verbatim}[commandchars=\\\{\}]
{\color{incolor}In [{\color{incolor}11}]:} \PY{o}{*}\PY{n}{newObjectPtr}
\end{Verbatim}


    \begin{Verbatim}[commandchars=\\\{\}]
(vectorCF212 \&) @0x4c32af0

    \end{Verbatim}

    \begin{Verbatim}[commandchars=\\\{\}]
{\color{incolor}In [{\color{incolor}12}]:} \PY{n}{newObjectPtr}\PY{o}{\PYZhy{}}\PY{o}{\PYZgt{}}\PY{n}{set}\PY{p}{(}\PY{l+m+mi}{5}\PY{p}{,}\PY{l+m+mi}{10}\PY{p}{)}\PY{p}{;}
\end{Verbatim}


    \begin{Verbatim}[commandchars=\\\{\}]
{\color{incolor}In [{\color{incolor}13}]:} \PY{n}{newObjectPtr}\PY{o}{\PYZhy{}}\PY{o}{\PYZgt{}}\PY{n}{get}\PY{p}{(}\PY{l+m+mi}{5}\PY{p}{)}
\end{Verbatim}


    \begin{Verbatim}[commandchars=\\\{\}]
(double) 10.000000

    \end{Verbatim}

    \begin{Verbatim}[commandchars=\\\{\}]
{\color{incolor}In [{\color{incolor}14}]:} \PY{k}{for}\PY{p}{(}\PY{k+kt}{int} \PY{n}{i} \PY{o}{=} \PY{l+m+mi}{0}\PY{p}{;} \PY{n}{i}\PY{o}{\PYZlt{}} \PY{n}{newObjectPtr}\PY{o}{\PYZhy{}}\PY{o}{\PYZgt{}}\PY{n}{size}\PY{p}{(}\PY{p}{)}\PY{p}{;}\PY{n}{i}\PY{o}{+}\PY{o}{+}\PY{p}{)}
             \PY{n}{cout}\PY{o}{\PYZlt{}}\PY{o}{\PYZlt{}}\PY{n}{newObjectPtr}\PY{o}{\PYZhy{}}\PY{o}{\PYZgt{}}\PY{n}{get}\PY{p}{(}\PY{n}{i}\PY{p}{)}\PY{o}{\PYZlt{}}\PY{o}{\PYZlt{}}\PY{l+s}{\PYZdq{}}\PY{l+s}{ }\PY{l+s}{\PYZdq{}}\PY{p}{;}
\end{Verbatim}


    \begin{Verbatim}[commandchars=\\\{\}]
0 0 0 0 0 10 0 0 0 0 
    \end{Verbatim}

    \begin{Verbatim}[commandchars=\\\{\}]
{\color{incolor}In [{\color{incolor}15}]:} \PY{k}{delete} \PY{n}{newObjectPtr}\PY{p}{;}
\end{Verbatim}


    \hypertarget{cux1ea5p-phuxe1t-bux1ed9-nhux1edb-ux111ux1ed9ng-cho-ux111ux1ed1i-tux1b0ux1ee3ng}{%
\subsubsection{Cấp phát bộ nhớ động cho đối
tượng}\label{cux1ea5p-phuxe1t-bux1ed9-nhux1edb-ux111ux1ed9ng-cho-ux111ux1ed1i-tux1b0ux1ee3ng}}

    \begin{Verbatim}[commandchars=\\\{\}]
{\color{incolor}In [{\color{incolor}16}]:} \PY{k}{class} \PY{n+nc}{NhanVien}
         \PY{p}{\PYZob{}}
            \PY{k}{public}\PY{o}{:}
               \PY{n}{NhanVien}\PY{p}{(}\PY{p}{)} \PY{p}{\PYZob{}} 
                  \PY{n}{cout} \PY{o}{\PYZlt{}}\PY{o}{\PYZlt{}} \PY{l+s}{\PYZdq{}}\PY{l+s}{Constructor duoc goi!}\PY{l+s}{\PYZdq{}} \PY{o}{\PYZlt{}}\PY{o}{\PYZlt{}}\PY{n}{endl}\PY{p}{;} 
                  
               \PY{p}{\PYZcb{}}
               \PY{o}{\PYZti{}}\PY{n}{NhanVien}\PY{p}{(}\PY{p}{)} \PY{p}{\PYZob{}} 
                  \PY{n}{cout} \PY{o}{\PYZlt{}}\PY{o}{\PYZlt{}} \PY{l+s}{\PYZdq{}}\PY{l+s}{Destructor duoc goi!}\PY{l+s}{\PYZdq{}} \PY{o}{\PYZlt{}}\PY{o}{\PYZlt{}}\PY{n}{endl}\PY{p}{;} 
                  
               \PY{p}{\PYZcb{}}
         \PY{p}{\PYZcb{}}\PY{p}{;}
\end{Verbatim}


    \begin{Verbatim}[commandchars=\\\{\}]
{\color{incolor}In [{\color{incolor}17}]:} \PY{n}{NhanVien}\PY{o}{*} \PY{n}{mangNhanVien} \PY{o}{=} \PY{k}{new} \PY{n}{NhanVien}\PY{p}{[}\PY{l+m+mi}{5}\PY{p}{]}\PY{p}{;}
         \PY{k}{delete} \PY{p}{[}\PY{p}{]} \PY{n}{mangNhanVien}\PY{p}{;} \PY{c+c1}{// xoa mang}
\end{Verbatim}


    \begin{Verbatim}[commandchars=\\\{\}]
Constructor duoc goi!
Constructor duoc goi!
Constructor duoc goi!
Constructor duoc goi!
Constructor duoc goi!
Destructor duoc goi!
Destructor duoc goi!
Destructor duoc goi!
Destructor duoc goi!
Destructor duoc goi!

    \end{Verbatim}


    % Add a bibliography block to the postdoc
    
    
    
    \end{document}
